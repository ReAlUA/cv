% https://ctan.org/pkg/moderncv?lang=en
\documentclass[11pt,a4paper]{moderncv}
\moderncvtheme[black]{classic}
\usepackage[scale=0.8, top=2cm, bottom=2cm]{geometry}

% Show/hide some details
\provideboolean{addrBrovary}
\setboolean{addrBrovary}{true}
%
\provideboolean{showLi}
\setboolean{showLi}{true}
%
\provideboolean{linkToLectures}
\setboolean{linkToLectures}{true}
%
\provideboolean{showPatentsInline}
\setboolean{showPatentsInline}{true}
%
\provideboolean{showPatentsSection}
\setboolean{showPatentsSection}{false}
%
\provideboolean{showOld}
\setboolean{showOld}{true}

% Personal data
\firstname{Oleksandr}
\familyname{Redchuk}
\title{Software Engineer}

\ifthenelse{\boolean{addrBrovary}}{%
\address{Ukraine, Brovary (Kyiv rgn)}
}{%
\address{Ukraine, Kyiv}
}%
\mobile{+38 096 4858089}
\email{oleksandr.redchuk@gmail.com}
\photo[64pt][0pt]{images/photo1}
\ifthenelse{\boolean{showLi}}{%
  \social[linkedin]{oleksandr-redchuk-7a8451165}
  \social[stackoverflow]{8848476/real}
  %\extrainfo{\protect\httpslink[real.kiev.ua]{real.kiev.ua}}
}{}
\social[github]{ReAlUA}

% To show numerical labels in the bibliography; only useful if you make
% citations in your resume
\makeatletter
\renewcommand*{\bibliographyitemlabel}{\@biblabel{\arabic{enumiv}}}
\makeatother
%
\definecolor{web}{rgb}{0.2,0.2,0.2}

% Content
\begin{document}

% Remove "Bibliography" title in "thebibliography" block
\renewcommand*{\bibliographyhead}[1]{}

\maketitle

\section{Summary}
Embedded software/hardware engineer with more than 30 years of professional experience.
Main industries are medical and automotive equipment, public safety (fire detection and alarm).
The main area of expertise is Linux kernel, RTOS, bare-metal firmware on ARM-based devices
as well as 8-bit microcontrollers.
For the last 5 years has been mostly working on downstreamed kernels (driver development and customization) and
RTOS-based firmware on Cortex-A (32-bit) + \mbox{Cortex-M} heterogeneous SoC.
Has prior background in design of embedded devices including SW and HW: analog and digital electronics,
algorithms, Bare-Metal and RTOS-based firmware development.

\section{Experience}

\cventry{2021--Present}{Software Engineer}{Autotalks}{}{}
  {Projects: Automotive V2X devices.\\
    Role: Driver development in Linux kernel, RTOS-based firmware and bootloader development
    Techologies: Cortex-A, Cortex-M; C, Assembler;
    \begin{itemize}
      \item Custom kernel driver development
      \item Adding new features to existing kernel drivers
      \item Migrating to the new kernel version
      \item Firmware development
      \item Bug fixing (drivers, device tree, frimware)
      \item Hardware debugging
    \end{itemize}}

\cventry{2020--2021}{Software Engineer}{GlobalLogic}{Kyiv}{}
  {Project: Medical IoT device\\
    Role: Leading a small team, SW\kern-0.2em/HW development\\
    Technologies: ESP32, FreeRTOS
    \begin{itemize}
      \item Development and optimization of algorithms
      \item Development of data acqusition and processing software
      \item Hardware debugging, schematic fixing and optimization
    \end{itemize}}

\cventry{2018--2021}{Software Engineer}{GlobalLogic}{Kyiv}{}
  {Project: Automotive device.\\
    Role: Driver development in Linux kernel, firmware development
    \begin{itemize}
      \item Custom WiFi driver implementing and speed optimization
      \item Hardware encryption accelerator driver speed optimization
      \item Adding new features to standard kernel drivers
      \item Firmware development (Cortex-M)
      \item Bug fixing (drivers, firmware)
    \end{itemize}}

\iftrue{% Shorter version
\cventry{2006--2018}{Head of SW\kern-0.2em/HW development}{Ista-Sital}{Kyiv}{}
  {Projects: EN54-compliant addressable fire detection and alarm system components.\\
    Role: Development of software and hardware (schematics, supervising of PCB development)\\
    Technologies: STM32L0, STM32F1, AVR; C, Assembler
    \begin{itemize}
      \item System-level design
      \item Creation of own loop-powered addressable devices protocol
      \item Smoke and heat detectors, manual call point and IO modules
      \item Loop controller for testing purposes
      \item Certification support
    \end{itemize}}
}\else{% More details
\cventry{2006--2018}{Head of SW\kern-0.2em/HW development}{Ista-Sital}{Kyiv}{}
  {Projects: EN54-compliant addressable fire detection and alarm system components.\\
    Patents: Registered design patent №14301 "Optical-electronic fire smoke detector" (Ukraine) \\
    Role: Development of software and hardware (schematics, supervising of PCB development)\\
    Technologies: STM32L0, STM32F1, AVR; C, Assembler
    \begin{itemize}
      \item System-level design
      \item Creation of own loop-powered addressable devices protocol
      \item Smoke and heat detectors, manual call point and IO modules
      \item Loop controller for testing/technology purposes
      \item Participation in the creation of a smoke/heat test tunnel
      \item Certification support
    \end{itemize}}
}\fi

\cventry{2015--2017}{SW\kern-0.2em/HW engineer (part time)}{National Aviation University}{Kyiv}{}
  {Project: Dedicated dead reckoning module — GPS emulator for Ardupilot mega.\\
    Role: SW\kern-0.2em/HW development
    \begin{itemize}
      \item FORTRAN to C++ code porting
      \item Test-vector-based PC verification.
      \item Implementation in microcontroller (STM32F303)
    \end{itemize}}

\cventry{1999--2006}{Head of Electronic Systems Department}{Teleoptic}{Kyiv}{}
  {Project: Line of multi-sensor medical digital X-ray detectors.\\
\ifthenelse{\boolean{showPatentsInline}}{%
Patents: UA77289C2 "X-ray receiver" (Ukraine), US7496177B2 "X-ray converter" (USA), DE602005013577D1 (Germany).  Utility model patnet UA1282U "Device for printing multi-format images on photosensitive film" (Ukraine).\\
}{}
    Role: SW\kern-0.2em/HW development\\
    Technologies: Altera MAX3K, Cyclone, AVR, LPC17xx; C, Assembler, AHDL, Verilog
    \begin{itemize}
      \item CCD cameras and data acquisition system (AFE and digital part) hardware design.
      \item Non-standard CCD timing design.
      \item PC to hardware control protocol design
      \item Embedded software (microcontrollers and programmable logic) development
      \item Windows 98/NT/XP control/data communication DLL
    \end{itemize}}

\ifthenelse{\boolean{showOld}}{%
  \cventry{1995-2005}{SW\kern-0.2em/HW engineer (part time)}{Ista-Sital}{Kyiv}{}{Main projects:
    \begin{itemize}
      \item Monitoring and video surveillance system.
      \item Electronic taximeter Pulsar-U with radio-modem interface to the local payment system.
      \item Interfaces, data loggers for security systems.
    \end{itemize}
    \ifthenelse{\boolean{showPatentsInline}}{Patent: UA37291 "Monitoring and video surveillance system" (Ukraine).\\}{}
    Role: SW\kern-0.2em/HW development\\
    Technologies: Altera FLEX8K, MCS-51, PIC16, AVR; C, Assembler, AHDL
    \begin{itemize}
      \item Schematics development
      \item Supervising PCB development
      \item Embedded software and programmable logic firmware development
      \item Control library development
    \end{itemize}}
  \cventry{1990-2005}{Research fellow}{Kyiv Polytechnic Institute, College of Instrument Design and Engineering}{Kyiv}{}{Main projects:
    \begin{itemize}
      \item Linear CCD-based portable spectrometers
      \item Matrix CCD-based spectrometer for aerial photography
      \item LWIR video camera
      \item ISA video capture board
    \end{itemize}
    Role: SW\kern-0.2em/HW development\\
    Technologies: Altera small PLD, EP10K, AVR; C, Assembler, PLDasm, AHDL
    \begin{itemize}
      \item Schematics and PCB development
      \item Embedded software and Control libraries development
    \end{itemize}}
  \cvitem{1985--1990}{\textit{Some interesting but too outdated projects}}%\\
}{%
  \cvitem{1985--1999}{\textit{Many interesting but outdated projects}}%\\
}%

\section{Education}
\cventry{1980--1985}{Master's degree}
  {Kyiv National University}{Kyiv}{}
  {\textit{Speciality}: Radio physics and electronics}

\section{Computer skills}
\cvdoubleitem{Languages}{C, Bash}{Projects}{Linux kernel, RTOS, Bare Metal}
\cvdoubleitem{CPUs}{ARM64, ARM32, x86}{MCUs}{STM32, LPC17xx, AVR, MCS51}
\cvdoubleitem{Tools}{GCC, GDB, Make, Git, Vim}{HW Tools}{scope, logical
		     analyzer, JTAG}
% Should add HDL in this case, so should differentiate knowlege level
%\cvitem{Hardware}{Altera FLEX, Cyclone; ADC/DAC/etc}

% \pagebreak

\section{Languages}
\cvlanguage{Ukrainian}{Fluent}{My native language}
\iffalse{%
\cvlanguage{Russian}{Fluent}{My native language}
}\fi
\cvlanguage{English}{Intermediate}{Speaking, reading and writing}

\pagebreak
\section{Other Activities}
\cvitem{2017--Today}{Translation of techical books from English to Ukrainian
    \begin{itemize}
      \item \textit{CLRS 3rd ed.} (translator, \(\sim \)1/4 of the book's content)%%\cite{CLRS}
      \item \textit{"Python crash course"} (co-editor)
      \item A new project is currently in progress
    \end{itemize}}
\vskip-1em % compensate extra space after itemize
\cvitem{2017--Today}{Member on StackOverflow (>1k reputation) \cite{stackoverflow}}
\cvitem{2006--Today}{Active Wikipedia editor -- more than 19,000 edits (Ukrainian language edition mostly)}
\cvitem{2020--2021}{Teacher on Linux kernel courses \ifthenelse{\boolean{linkToLectures}}{\cite{glkt2020}}{(GlobalLogic)}
}
\cvitem{1998--2018}{AVReAl -- AVR ISP programming software for MS-DOS, Windows,
Linux and FreeBSD \cite{avreal}}

%% \ifthenelse{\boolean{showPatentsSection}}{%
%% \section{Patents}
%% \begin{itemize}
%% \item
%% \end{itemize}}{}%

\section{References}

% TODO: Make interval smaller
\begin{thebibliography}{9}
%% \ifthenelse{\boolean{showPatentsSection}}{%
%% \bibitem{} {\color{web} \url{}}
%% }{}%
%%\bibitem{CLRS}{\color{web}\url{https://uk.wikipedia.org/wiki/CLRS}}
\bibitem{stackoverflow}{\color{web}\url{https://stackoverflow.com/users/8848476/real}}
\ifthenelse{\boolean{linkToLectures}}{%
\bibitem{glkt2020} {\color{web} \url{https://github.com/ReAlUA/kernel-lectures}}
}{}%
%\bibitem{glkt2021} {\color{web} \url{https://github.com/Kernel-GL-HRK/gl-kernel-training-2021}}
\bibitem{avreal} {\color{web} \url{https://real.kyiv.ua/avreal/}}
\end{thebibliography}

\end{document}


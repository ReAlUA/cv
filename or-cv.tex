% https://ctan.org/pkg/moderncv?lang=en
\documentclass[11pt,a4paper]{moderncv}
\moderncvtheme[black]{classic}
\usepackage[scale=0.8]{geometry}

% Hide some details
\provideboolean{forAbroad}
\setboolean{forAbroad}{false}

% Personal data
\firstname{Oleksandr}
\familyname{Redchuk}
\title{Software Engineer}

\ifthenelse{\boolean{forAbroad}}{%
\address{Ukraine, Kyiv}
}{%
\address{Ukraine, Brovary (Kyiv rgn)}
}%
\mobile{+38 096 4858089}
\email{oleksandr.redchuk@gmail.com}
\photo[64pt][0pt]{images/photo1}
\social[github]{ReAlUA}
%\social[linkedin]{oleksandr-redchuk-7a8451165}

% To show numerical labels in the bibliography; only useful if you make
% citations in your resume
\makeatletter
\renewcommand*{\bibliographyitemlabel}{\@biblabel{\arabic{enumiv}}}
\makeatother
%
\definecolor{web}{rgb}{0.2,0.2,0.2}

% Content
\begin{document}

% Remove "Bibliography" title in "thebibliography" block
\renewcommand*{\bibliographyhead}[1]{}

\maketitle

\section{Summary}
Embedded software/hardware engineer with more than 25 years of professional experience.
Main industries are medical and automotive equipment, public safety (fire detection and alarm).
The main area of expertise is Linux kernel, RTOS, firmwares on ARM-based devices
as well as 8-bit microcontrollers.
For the last 4 years has been mostly working on downstreamed kernels (driver development and customization) and
RTOS-based firmware on Cortex-A (32-bit) + Cortex-M heterogeneous SoC.
Has prior background in design of embedded devices including SW and HW: electronics, algorithms,
Bare-Metal and RTOS-based firmware development.

\section{Experience}
\cventry{2021--Present}{Software Engineer}{Auto-talks}{}{}
  {Projects: Automotive V2X devices.\\
    Role: Driver development in Linux kernel, firmware development
    \begin{itemize}
      \item Custom kernel driver development
      \item Adding new features to standard kernel drivers
      \item Migrating to new kernel version
      \item Bug fixing (drivers, device tree, frimware)
      \item Hardware debugging
    \end{itemize}}
\cventry{2020--2021}{Software Engineer}{GlobalLogic}{Kyiv}{}
  {Project: Medical IoT device (ESP32, FreeRTOS)\\
    Role: Leading a small team, SW/HW development.
    \begin{itemize}
      \item Development and optimization of algorithms
      \item Developing data acqusition and processing software
      \item Hardware debugging, schematic fixing and optimization
    \end{itemize}}
\cventry{2018--2021}{Software Engineer}{GlobalLogic}{Kyiv}{}
  {Project: Automotive device.\\
    Role: Driver development in Linux kernel, firmware development
    \begin{itemize}
      \item Custom WiFi driver implementing and speed optimization
      \item Hardware encryption accelerator driver speed optimization
      \item Adding new features to standard kernel drivers
      \item Firmware development (Cortex-M)
      \item Bug fixing (drivers, device tree)
    \end{itemize}}
\cventry{2006--2018}{Chief SW/HW engineer}{Ista-Sital}{Kyiv}{}
  {Projects: Fire detection and I/O devices.\\
    Role: Driver development in Linux kernel, firmware development
    \begin{itemize}
      \item Custom WiFi driver implementing and speed optimization
      \item Hardware encryption accelerator driver speed optimization
      \item Adding new features to standard kernel drivers
      \item Firmware development (Cortex-M)
      \item Bug fixing (drivers, device tree)
    \end{itemize}}
\cventry{2015--2017}{SW/HW engineer (part time)}{National Aviation University}{Kyiv}{}
  {Project: Dedicated dead reckoning module — GPS emulator for Ardupilot mega.\\
    Role: SW/HW development
    \begin{itemize}
      \item FORTRAN to C++ code porting
      \item test-vector-based PC verification.
      \item Implementation in microcontroller (STM32F303)
    \end{itemize}}
\cventry{1999--2006}{Head of Electronic Systems Department}{Teleoptic}{Kyiv}{}
  {Project: Line of multi-sensor medical digital X-ray detectors (international patent
WO2006049589A1 and Ukraine patent)\\
    Role: SW/HW development
    \begin{itemize}
      \item CCD cameras and data acquisition system (AFE and digital part) hardware design.
      \item Non-standard CCD timing design.
      \item PC to hardware control protocol design
      \item Embedded software (microcontrollers and programmable logic) development
      \item Windows 98/NT/XP control/data communication DLL
    \end{itemize}}

\cvitem{1985--1999}{\textit{Many interesting but outdated projects}%\\
% See details on LnkedIn \cite{c} % TODO: add it
}

\section{Education}
\cventry{1980--1985}{Master's degree}
  {Kyiv National University}{Kyiv}{}
  {\textit{Speciality}: Radio physics and electronics}

\section{Computer skills}
\cvdoubleitem{Languages}{C, Bash}{Projects}{Linux kernel, RTOS, Bare Metal}
\cvdoubleitem{CPUs}{ARM32, x86}{MCUs}{STM32, LPC17xx, AVR, MCS51}
\cvdoubleitem{Tools}{GCC, GDB, Make, Git, Vim}{HW Tools}{scope, logical
		     analyzer, JTAG}
% Should add HDL in this case, so should differentiate knowlege level
%\cvitem{Hardware}{Altera FLEX, Cyclone; ADC/DAC/etc}

% \pagebreak

\section{Languages}
\cvlanguage{Ukrainian}{Fluent}{My native language}
\iffalse{%
\cvlanguage{Russian}{Fluent}{My native language}
}\fi
\cvlanguage{English}{Intermediate}{Speaking, reading and writing}

\section{Other Activities}

\cvitem{2017--Today}{Translation of techical books from English to Ukrainian
    \begin{itemize}
      \item \textit{CLRS 3rd ed.} (translator)
      \item \textit{"Python crash course"} (editor)
      \item new projects --- work in progress...
    \end{itemize}}
\vskip-1em % compensate extra space after itemize
\cvitem{2017--Today}{Member on StackOverflow (>1k reputation) \cite{stackoverflow}}
\cvitem{2020--2021}{Instructor of Linux kernel courses}
\cvitem{1998--2018}{Creator of AVReAl -- AVR ISP programming software for MS-DOS, Windows,
Linux and FreeBSD \cite{avreal}}

\section{References}

% TODO: Make smalle interval
\begin{thebibliography}{9}
\bibitem{stackoverflow} {\color{web} \url{https://stackoverflow.com/users/8848476/real}}
%\bibitem{glkt2021} {\color{web} \url{https://github.com/Kernel-GL-HRK/gl-kernel-training-2021}}
\bibitem{avreal} {\color{web} \url{https://real.kyiv.ua/avreal/}}
\end{thebibliography}

\end{document}

